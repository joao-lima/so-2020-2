
\documentclass[xcolor=dvipsnames,11pt,presentation,aspectratio=169]{beamer}
\usepackage[utf8]{inputenc}
\usepackage[brazilian]{babel}
\usepackage{verbatim}
\usepackage{graphicx}
\usepackage{xspace}
\usepackage{color}
\usepackage{amsthm}
\usepackage{url}
\usepackage{array}
\usepackage{hyperref}
\usepackage{times,mathptmx}
\usepackage{multimedia}
\usepackage{framed}

\usetheme{Madrid}

\setbeamercovered{transparent}
\setbeamertemplate{footline}[frame number]
\usecolortheme[named=BrickRed]{structure}
\setbeamertemplate{navigation symbols}{}
%\setbeamertemplate{footline}[frame number]
%\setbeamersize{text margin left=1em,text margin right=1em}
\newcommand{\bashcmd}[1]{\textcolor{White}{\colorbox{Sepia}{\texttt{#1}}}}

\newcommand{\disciplina}{Sistemas Operacionais}
\newcommand{\aula}{Introdução POSIX Threads}
\newcommand{\nome}{João Vicente Ferreira Lima}

\lecture[1]{\aula}{aula02}
\def\lecturename{\aula}

\newcommand{\Red}[1]{{\color{red}#1}}
\newcommand{\red}[1]{{\color{red}#1}}
\newcommand{\Blue}[1]{{\color{blue}#1}}
\newcommand{\blue}[1]{{\color{blue}#1}}

\newcommand{\PBS}[1]{\let\temp=\\#1\let\\=\temp}
\newcommand{\RRCOL}{\PBS\raggedright\hspace{0pt}}

\title[\aula]{\aula}

\subtitle{\disciplina}

\author[João Vicente Ferreira Lima]{\nome}

\institute[UFSM]{Departamento de Linguagens e Sistemas de Computação \\ Universidade Federal de Santa Maria \\ \url{jvlima@inf.ufsm.br}}
\date{2020/2}

\graphicspath{{.}{figs/}{../logos/}}

\logo{
    \includegraphics[height=1cm,width=1cm,keepaspectratio]{logo_inf}
    \includegraphics[height=1cm,width=1cm,keepaspectratio]{logo_ufsm}
}

\newtheorem{mydef}{Definição}[section]
%\newtheorem{myteo}{Teorema}[section]

% Typesetting Listings
\usepackage{listings}
\lstset{%
language=C,
basicstyle=\footnotesize\sffamily,
commentstyle=\slshape\color{green!50!black},
keywordstyle=\bfseries\color{blue!50!black},
stringstyle=\color{red},
numbers=none,
% escapechar=\#,
breaklines=false,
emphstyle=\color{red}
}

\begin{document}

\begin{frame}
  \maketitle
\end{frame}

%%%%%%%%%%%%%%%%%%%%%%%%%%%%%%%%%%%%%%%%%%%%%%%%%%%%%%%%%%%%%%%%%%%%%%%%%%%%%%%

\frame{
    \frametitle<presentation>{Sumário}
    \tableofcontents[hideallsubsections]
}

\AtBeginSection{
  \begin{frame}
    \frametitle{Sumário}
    \tableofcontents[currentsection,hideothersubsections]
  \end{frame}
}

\AtBeginSubsection{
  \begin{frame}
    \frametitle{Sumário}
    \tableofcontents[currentsection,currentsubsection]
  \end{frame}
}

%%%%%%%%%%%%%%%%%%%%%%%%%%%%%%%%%%%%%%%%%%%%%%%%%%%%%%%%%%%%%%%%%%%%%%%%%%%%%%%
\section{POSIX Threads}
%%%%%%%%%%%%%%%%%%%%%%%%%%%%%%%%%%%%%%%%%%%%%%%%%%%%%%%%%%%%%%%%%%%%%%%%%%%%%%%
%-----------------------------------------------------------------------------%
%\subsection{Introdução}
%-----------------------------------------------------------------------------%
\begin{frame}
  \frametitle{POSIX Threads}
  \begin{itemize}
  \item Padrão POSIX para threads definido em 1995 em C.
  \item Implementado por diversos sistemas UNIX-Like:
    \begin{itemize}
      \item FreeBSD, NetBSD, OpenBSD, Linux, MacOS, Android, Solaris, etc.
      \end{itemize}
  \item Categorizado em 4 grupos:
    \begin{itemize}
      \item Gerenciamento de threads
      \item Mutexes
      \item Variáveis de condição
      \item Sincronização
    \end{itemize}      
  \end{itemize}
\end{frame}
%%%%%%%%%%%%%%%%%%%%%%%%%%%%%%%%%%%%%%%%%%%%%%%%%%%%%%%%%%%%
\begin{frame}
  \frametitle{Criando threads}
  \begin{itemize}
  \item Principais funções
    \begin{itemize}
      \item \texttt{pthread\_create}
      \item \texttt{pthread\_exit}
      \item \texttt{pthread\_join}
      \item \texttt{pthread\_yield}
      \item \texttt{pthread\_cancel}
    \end{itemize}      
  \end{itemize}
\end{frame}
%%%%%%%%%%%%%%%%%%%%%%%%%%%%%%%%%%%%%%%%%%%%%%%%%%%%%%%%%%%%
\begin{frame}[fragile]
  \frametitle{Criando threads}
\begin{lstlisting}
#include <pthread.h>
#include <stdio.h>
#include <stdlib.h>
#define NUM_THREADS	5

int main(int argc, char *argv[])
{
   pthread_t threads[NUM_THREADS];
   int rc;
   long t;
   for( t= 0; t < NUM_THREADS; t++ ) {
     rc = pthread_create( &threads[t], NULL, PrintHello,
                         (void *)t );
     // testa rc
   }

   pthread_exit(NULL);  // return 0
}
\end{lstlisting}
\end{frame}

% TODO exemplo join
% 
%%%%%%%%%%%%%%%%%%%%%%%%%%%%%%%%%%%%%%%%%%%%%%%%%%%%%%%%%%%%
\begin{frame}[fragile]
  \frametitle{Criando threads}
\begin{lstlisting}

void *PrintHello(void *threadid)
{
   long tid;
   tid = (long)threadid;
   printf("Hello World! It's me, thread #%ld!\n", tid);
   pthread_exit(NULL);
 }

\end{lstlisting}
\end{frame}
%%%%%%%%%%%%%%%%%%%%%%%%%%%%%%%%%%%%%%%%%%%%%%%%%%%%%%%%%%%%
\begin{frame}[fragile]
  \frametitle{Criar e esperar}
\begin{lstlisting}
int main (int argc, char *argv[])
{
   pthread_t thread[NUM_THREADS];
   long t;
   void *status;

   for(t=0; t<NUM_THREADS; t++) {
     pthread_create(&thread[t], NULL, PrintHello,
         (void *)t); 
   }

   for(t=0; t<NUM_THREADS; t++)
       pthread_join(thread[t], &status);

   pthread_exit(NULL);
}
\end{lstlisting}
\end{frame}


%%%%%%%%%%%%%%%%%%%%%%%%%%%%%%%%%%%%%%%%%%%%%%%%%%%%%%%%%%%%%%%%%%%%%%%%%%%%%%%
\section{Mutexes}
%%%%%%%%%%%%%%%%%%%%%%%%%%%%%%%%%%%%%%%%%%%%%%%%%%%%%%%%%%%%%%%%%%%%%%%%%%%%%%%

\begin{frame}
  \frametitle{Mutexes}
  \begin{itemize}
  \item São variáveis de trava do tipo dormir/acordar
  \item Variações são de testar, ou travas leituras/escrita
  \item Funções:
    \begin{itemize}
      \item \texttt{pthread\_mutex\_init}
      \item \texttt{pthread\_mutex\_destroy}
      \item \texttt{pthread\_mutex\_lock}
      \item \texttt{pthread\_mutex\_unlock}
      \item \texttt{pthread\_mutex\_trylock}
    \end{itemize}
  \end{itemize}
\end{frame}
%%%%%%%%%%%%%%%%%%%%%%%%%%%%%%%%%%%%%%%%%%%%%%%%%%%%%%%%%%%%
\begin{frame}[fragile]
  \frametitle{Exemplo mutex}
  \vspace{-5mm}
\begin{lstlisting}
#include <stdio.h>
#include <pthread.h>

pthread_mutex_t m; // lock
int main(void ) {
    pthread_t tid1, tid2;
    
    pthread_mutex_init(&m, NULL);

    pthread_create(&tid1, NULL, countgold, NULL);
    pthread_create(&tid2, NULL, countgold, NULL);
    // espera as duas threads
    pthread_join(tid1, NULL);
    pthread_join(tid2, NULL);

    pthread_mutex_destroy(&m);
    printf("The sum is %d\n", sum);

    return 0;
}
\end{lstlisting}
\end{frame}
%%%%%%%%%%%%%%%%%%%%%%%%%%%%%%%%%%%%%%%%%%%%%%%%%%%%%%%%%%%%
\begin{frame}[fragile]
  \frametitle{Exemplo mutex}
\begin{lstlisting}
int sum = 0; // global

void *countgold(void *param) {
    int i;

    // regiao critica
    pthread_mutex_lock(&m);

    for (i = 0; i < 10000000; i++) {
	    sum += 1;
    }
      
    pthread_mutex_unlock(&m);
    return NULL;
}
\end{lstlisting}
\end{frame}

%%%%%%%%%%%%%%%%%%%%%%%%%%%%%%%%%%%%%%%%%%%%%%%%%%%%%%%%%%%%%%%%%%%%%%%%%%%%%%%
\section{Variáveis de condição}
%%%%%%%%%%%%%%%%%%%%%%%%%%%%%%%%%%%%%%%%%%%%%%%%%%%%%%%%%%%%%%%%%%%%%%%%%%%%%%%
\begin{frame}
  \frametitle{Variáveis de condição}
  \begin{itemize}
  \item Variáveis que permitem bloquear/esperar um evento
  \item Sinalizar uma thread ou várias
  \item Funções importantes:
  \begin{itemize}
    \item \texttt{pthread\_cond\_wait}
    \item \texttt{pthread\_cond\_signal}
    \item \texttt{pthread\_cond\_broadcast}
  \end{itemize}
  \end{itemize}
\end{frame}
%%%%%%%%%%%%%%%%%%%%%%%%%%%%%%%%%%%%%%%%%%%%%%%%%%%%%%%%%%%%%%%%%%%%%%%%%%%%%%%
\begin{frame}[fragile]
  \frametitle{Variáveis de condição}
  \vspace{-5mm}
\begin{lstlisting}
#include <pthread.h>
#include <stdio.h>
#include <stdlib.h>
  
#define NUM_THREADS  3
#define TCOUNT 10
#define COUNT_LIMIT 12
  
int     count = 0;
pthread_mutex_t count_mutex;
pthread_cond_t count_threshold_cv;  
\end{lstlisting}
\end{frame}
%%%%%%%%%%%%%%%%%%%%%%%%%%%%%%%%%%%%%%%%%%%%%%%%%%%%%%%%%%%%
\begin{frame}[fragile]
  \frametitle{Variáveis de condição}
  \vspace{-5mm}
\begin{lstlisting}
int main(int argc, char *argv[])
{
  int i, rc; 
  long t1=1, t2=2, t3=3;
  pthread_t threads[3];
  pthread_mutex_init(&count_mutex, NULL);
  pthread_cond_init (&count_threshold_cv, NULL);

  pthread_create(&threads[0], NULL, watch_count, (void *)t1);
  pthread_create(&threads[1], NULL, inc_count, (void *)t2);
  pthread_create(&threads[2], NULL, inc_count, (void *)t3);
  for (i = 0; i < NUM_THREADS; i++) 
    pthread_join(threads[i], NULL);

  printf ("Joined with %d threads. Final value = %d. Done.\n", 
     NUM_THREADS, count);
  
  pthread_mutex_destroy(&count_mutex);
  pthread_cond_destroy(&count_threshold_cv);
  pthread_exit (NULL);
}
\end{lstlisting}
\end{frame}
%%%%%%%%%%%%%%%%%%%%%%%%%%%%%%%%%%%%%%%%%%%%%%%%%%%%%%%%%%%%
\begin{frame}[fragile]
  \frametitle{Variáveis de condição}
  \vspace{-5mm}
\begin{lstlisting}
  void *inc_count(void *t) 
  {
    int i;
    long my_id = (long)t;
  
    for (i=0; i < TCOUNT; i++) {
      pthread_mutex_lock(&count_mutex);
      count++;
  
      if (count == COUNT_LIMIT)
        // limite: avisa thread que monitora
        pthread_cond_signal(&count_threshold_cv);

      pthread_mutex_unlock(&count_mutex);
      sleep(1);
    }
    pthread_exit(NULL);
  }
\end{lstlisting}
\end{frame}
%%%%%%%%%%%%%%%%%%%%%%%%%%%%%%%%%%%%%%%%%%%%%%%%%%%%%%%%%%%%
\begin{frame}[fragile]
  \frametitle{Variáveis de condição}
\begin{lstlisting}

void *watch_count(void *t) 
{
  long my_id = (long)t;

  pthread_mutex_lock(&count_mutex);
  while (count < COUNT_LIMIT) {
    // esperando sinal
    pthread_cond_wait(&count_threshold_cv, &count_mutex);
  }
  
  count += 125;

  pthread_mutex_unlock(&count_mutex);
  pthread_exit(NULL);
}

\end{lstlisting}
\end{frame}

%%%%%%%%%%%%%%%%%%%%%%%%%%%%%%%%%%%%%%%%%%%%%%%%%%%%%%%%%%%%%%%%%%%%%%%%%%%%%%%
\section{Sincronização}
%%%%%%%%%%%%%%%%%%%%%%%%%%%%%%%%%%%%%%%%%%%%%%%%%%%%%%%%%%%%%%%%%%%%%%%%%%%%%%%
\begin{frame}
  \frametitle{Sincronização}
  \begin{itemize}
  \item Pthreads suportam barreiras de Sincronização
  \item As thread esperam todas chegarem até certo ponto para continuar
  \item Funções:
    \begin{itemize}
      \item \texttt{pthread\_barrier\_init}
      \item \texttt{pthread\_barrier\_destroy}
      \item \texttt{pthread\_barrier\_wait}
    \end{itemize}
  \end{itemize}
\end{frame}
%%%%%%%%%%%%%%%%%%%%%%%%%%%%%%%%%%%%%%%%%%%%%%%%%%%%%%%%%%%%
\begin{frame}[fragile]
  \frametitle{Sincronização}
  \vspace{-5mm}
\begin{lstlisting}
  pthread_barrier_t barreira;

  int main(int argc, char *argv[])
  {
     pthread_t threads[NUM_THREADS];
     long t;

     pthread_barrier_init(&barreira, NULL, NUM_THREADS);

     for( t= 0; t < NUM_THREADS; t++ ) {
       rc = pthread_create( &threads[t], NULL, PrintHello,
                           (void *)t );
       // testa rc
     }

     pthread_barrier_destroy(&barreira);
     pthread_exit(NULL);  // return 0
  }
\end{lstlisting}
\end{frame}
%%%%%%%%%%%%%%%%%%%%%%%%%%%%%%%%%%%%%%%%%%%%%%%%%%%%%%%%%%%%
\begin{frame}[fragile]
  \frametitle{Sincronização}
  \vspace{-5mm}
\begin{lstlisting}

  void *PrintHello(void *threadid)
  {
     long tid;
     tid = (long)threadid;
     printf("Thread #%ld antes da barreira!\n", tid);

     pthread_barrier_wait(&barreira);

     printf("Thread #%ld passou!\n", tid);
     pthread_exit(NULL);
   }
  
\end{lstlisting}
\end{frame}
%%%%%%%%%%%%%%%%%%%%%%%%%%%%%%%%%%%%%%%%%%%%%%%%%%%%%%%%%%%%


%%%%%%%%%%%%%%%%%%%%%%%%%%%%%%%%%%%%%%%%%%%%%%%%%%%%%%%%%%%%%%%%%%%%%%%%%%%%%%%
\section{Thread Local Storage}
%%%%%%%%%%%%%%%%%%%%%%%%%%%%%%%%%%%%%%%%%%%%%%%%%%%%%%%%%%%%%%%%%%%%%%%%%%%%%%%
\begin{frame}
  \frametitle{Thread Local Storage}
  \begin{itemize}
  \item Pthreads permite ter dados globais mas específicos para cada thread.
  \item As funções são:
    \begin{itemize}
      \item \texttt{pthread\_key\_create}
      \item \texttt{pthread\_getspecific}
      \item \texttt{pthread\_setspecific}
      \item \texttt{pthread\_key\_delete}
    \end{itemize}  
  \end{itemize}
\end{frame}
%%%%%%%%%%%%%%%%%%%%%%%%%%%%%%%%%%%%%%%%%%%%%%%%%%%%%%%%%%%%
\begin{frame}[fragile]
  \frametitle{Exemplo TLS}
  \vspace{-5mm}
\begin{lstlisting}
static pthread_key_t   chave;
static pthread_once_t  inicia_once = PTHREAD_ONCE_INIT;
// esta e chamada apenas quando destroi o dado
void libera_no_fim (void *buffer){
  free(buffer);
}
// esta e chamada por apenas uma thread
void inicializa (void){
  pthread_key_create(&grader_key, libera_no_fim);
}
// varias threads chamam esta funcao
void *aloca(void *param) { 
  char* texto;
  pthread_once(&inicia_once, inicializa); // apenas uma thread 
  texto = pthread_getspecific(chave);
  if(texto == NULL){
    texto = (char*)malloc( sizeof(char)*100 );
    pthread_setspecific(chave, texto);
  }
  pthread_exit(NULL);
}
\end{lstlisting}
\end{frame}
%%%%%%%%%%%%%%%%%%%%%%%%%%%%%%%%%%%%%%%%%%%%%%%%%%%%%%%%%%%%
\begin{frame}[fragile]
  \frametitle{GCC TLS}
  \vspace{-5mm}
  Outro alternativa é o \texttt{\_\_thread} do C11.
\begin{lstlisting}
__thread char *texto = NULL;

// varias threads chamam esta funcao
void *aloca(void *param) { 
  char* texto;

  if(texto == NULL){
    texto = (char*)malloc( sizeof(char)*100 );
    // cada thread tem sua copia
  }

  pthread_exit(NULL);
}
\end{lstlisting}
\end{frame}
%%%%%%%%%%%%%%%%%%%%%%%%%%%%%%%%%%%%%%%%%%%%%%%%%%%%%%%%%%%%
\begin{frame}[fragile]
  \frametitle{GCC Atomics}
GCC possui funções built-in do compilador como abaixo:
\begin{lstlisting}
type __sync_fetch_and_add (type *ptr, type value, ...)
type __sync_fetch_and_sub (type *ptr, type value, ...)
type __sync_fetch_and_or (type *ptr, type value, ...)
type __sync_fetch_and_and (type *ptr, type value, ...)
type __sync_fetch_and_xor (type *ptr, type value, ...)
type __sync_fetch_and_nand (type *ptr, type value, ...)

type __sync_add_and_fetch (type *ptr, type value, ...)
type __sync_sub_and_fetch (type *ptr, type value, ...)
type __sync_or_and_fetch (type *ptr, type value, ...)
type __sync_and_and_fetch (type *ptr, type value, ...)
type __sync_xor_and_fetch (type *ptr, type value, ...)
type __sync_nand_and_fetch (type *ptr, type value, ...)
\end{lstlisting}
\end{frame}
%%%%%%%%%%%%%%%%%%%%%%%%%%%%%%%%%%%%%%%%%%%%%%%%%%%%%%%%%%%%
\begin{frame}[fragile]
  \frametitle{GCC Atomics}
GCC possui funções built-in do compilador como abaixo:
\begin{lstlisting}
  bool __sync_bool_compare_and_swap (type *ptr,
      type oldval type newval, ...)
  type __sync_val_compare_and_swap (type *ptr,
      type oldval type newval, ...)

  __sync_synchronize (...)

  void __sync_lock_release (type *ptr, ...)
\end{lstlisting}
\end{frame}
%%%%%%%%%%%%%%%%%%%%%%%%%%%%%%%%%%%%%%%%%%%%%%%%%%%%%%%%%%%%
\begin{frame}[fragile]
  \frametitle{Exemplo de soma com GCC atomics}
\begin{lstlisting}
int sum = 0; // global

void *countgold(void *param) {
    int i;

    // regiao critica
    for (i = 0; i < 10000000; i++)
      __sync_fetch_and_add( &sum, 1 ) ;
  

    return NULL;
}
\end{lstlisting}
\end{frame}


%%%%%%%%%%%%%%%%%%%%%%%%%%%%%%%%%%%%%%%%%%%%%%%%%%%%%%%%%%%%%%%%%%%%%%%%%%%%%%%
%\begin{frame}[plain]{}
%\end{frame}
%%%%%%%%%%%%%%%%%%%%%%%%%%%%%%%%%%%%%%%%%%%%%%%%%%%%%%%%%%%%%%%%%%%%%%%%%%%%%%%
\begin{frame}
  \frametitle{}
  \begin{center}
  \begin{columns}
    \begin{column}{0.5\textwidth}
    \raggedleft
\includegraphics[width=4cm]{logo_ufsm}
    \end{column}
    \begin{column}{0.5\textwidth}
\includegraphics[width=4cm]{logo_inf}
    \end{column}
  \end{columns}
\end{center}

\end{frame}
%%%%%%%%%%%%%%%%%%%%%%%%%%%%%%%%%%%%%%%%%%%%%%%%%%%%%%%%%%%%%%%%%%%%%%%%%%%%%%%

\end{document}
